\documentclass{article}

\usepackage{geometry}
\usepackage{graphicx}

% script F
\usepackage{mathrsfs}

% tight spacing on lists
\usepackage{enumitem}
\setlist{noitemsep}

% bibliography
\usepackage[round]{natbib} %bibliography citation style with round brackets

% colors
\usepackage[usenames,dvipsnames]{xcolor}
\definecolor{light-gray}{gray}{0.95}

% in-document hyper refs
\usepackage[colorlinks,citecolor=Black,linkcolor=Black,urlcolor=Black,filecolor=Black]{hyperref}

% in-text listing
\usepackage{listings}

% listings style
\lstset{
  backgroundcolor=\color{light-gray},
  basicstyle=\ttfamily\scriptsize,
  breakatwhitespace=false,
  breaklines=true,
  frame=single,
  keepspaces=true,
}

% SI units package
\usepackage{siunitx}
\sisetup{per-mode = symbol}

% algorithm
\usepackage{algorithm}
\usepackage{algpseudocode}

% line spacing
\linespread{1.3}

\title{\textbf{Monte-Carlo Ray Tracing} \\ Program Documentation}
\date{\today}
\author{Matt S. Mitchell}

\begin{document}

\maketitle
\clearpage
\tableofcontents
\clearpage

\section{Introduction}

Computation of the radiant energy exchange between surfaces within an enclosure often requires computation of view factors. For simplified geometries these view factors can be evaluated analytically or numerically. However, for more complicated geometries or for more complicated physical phenomena (e.g.~participating media or specular reflection), the view factors may need to be evaluated using the Monte-Carlo Method. In general, the Monte-Carlo method is a stochastic method which takes a model along with a set of potential input values, and then perturbs the input values randomly within a plausible range. From the model output, a probability distribution is generated which helps identify the most likely set of outcomes, given a random distribution of plausible input values.

\cite{Nigusse2004}, as part of a class project for MAE 5823, applied this method to radiation heat transfer and created the Monte-Carlo Ray Tracing (MCRT) program. The program is used to solve for the distribution factors, $D_{ij}$, which are related to the total exchange areas, $\overline{S_i S_j}$, by Equation \ref{eq:dist-to-totExchArea}

\begin{equation}
    D_{ij} = \frac{\overline{S_i S_j}}{\varepsilon_i A_i}
    \label{eq:dist-to-totExchArea}
\end{equation}

where, $\varepsilon_i$ is the surface emissivity and $A_i$ is the surface area. \cite{Modest2003}, and others commonly refer to this as the ``Script-F" view factor, $\mathscr{F}_{ij}$. With these distribution factors, users can apply the net-radiant exchange or zonal methods to solve for the radiant energy exchange within an enclosure.

A description of the methodology used to compute the distribution factors is given in Section \ref{sec:method}. Information regarding program input and output can be found in Sections \ref{sec:input} and \ref{sec:output}, respectively. Finally, recommended future work is described in Section \ref{sec:future work}.

\section{Methodology}
\label{sec:method}

For a full description of the methodology, see \citep{Nigusse2004}. In summary, the program assumes a gray enclosure. Using the input information defined in Section \ref{sec:input}, the program defines surface equations, normal vectors, tangent vectors, and surface areas for each surface. Then from each surface, $i$, the program emits a user-specified number of rays. Once the ray is absorbed by another surface, $j$, 1 is added to the distribution factor $D_{ij}$ for that surface. $D_{ij}$ is then normalized by the total number of surface rays emitted by surface $i$ at the end of the simulation. The direction the ray is emitted or reflected depends on the respective surface type. This is summarized in the following algorithm.

\begin{algorithm}
\caption{MCRT algorithm}
\begin{algorithmic}[1]
\Procedure{MCRT}{}
\State Read vertex and surface geometry and properties
\State Initialize simulation
\State Determine surface normal vectors, tangent vectors, surface equations, and surface areas
\For{$N_{surf}$}
\For{$N_{bundle}$}
\State Determine bundle emission point

\State Determine direction of emitted bundle

\State Determine bundle intersection point

\State Check for absorption or reflection
\EndFor
\EndFor
\State Compute distribution factors
\State Compute energy balance
\EndProcedure
\end{algorithmic}
\end{algorithm}

\section{Input}
\label{sec:input}

MCRT leverages the input format used by the View3D program \citep{Walton2002}. In general both  programs require a list of vertex locations and surface definitions. In addition to this, MCRT will also take a list of surface types which define how each surface behaves within the Monte-Carlo simulation. If the surface type is omitted, the program will assume default values of ``DIF" for each program, which is described in Section \ref{s-s-sec:dif}.

\subsection{Input Files}

\subsubsection{input.vs3}

The input.vs3 file will contain all of the information defined in Sections \ref{subsec:verts}--\ref{subsec:comments}. The file must be named ``input.vs3" and be located in the same directory as the MCRT executable. An example input.vs3 file is provide below.

\begin{lstlisting}
!   #   x   y   z
V   1   8   0   0
V   2  13   0   0
V   3  13   8   0
V   4   8  13   0
V   5   0  13   0
V   6   0   8   0
V   7   8   8   0
V   8   8   0   5
V   9  13   0   5
V  10  13   8   5
V  11   8  13   5
V  12   0  13   5
V  13   0   8   5
V  14   8   8   5
!   #  v1  v2  v3  v4 base cmb  emit   name      surface data
S   1   8   9   2   1   0   0   0.90   South-Wall-Bottom-(SDE)
S   2   9  10   3   2   0   0   0.90   East-Wall
S   3  10  11   4   3   0   0   0.99   Angle-Wall
S   4  11  12   5   4   0   0   0.90   North-Wall
S   5  12  13   6   5   0   0   0.90   West-Wall-Top
S   6  13  14   7   6   0   0   0.90   South-Wall-Top
S   7  14   8   1   7   0   0   0.90   West-Wall-Bottom
S   8   1   2   3   7   0   0   0.90   South-Floor
S   9   7   3   4   0   0   8   0.90   Angle-Floor
S  10   4   5   6   7   0   8   0.90   North-Floor
S  11  14  10   9   8   0   0   0.90   South-Ceiling
S  12  14  11  10   0   0   11  0.90   Angle-Ceiling
S  13  13  12  11  14   0   11  0.90   North-Ceiling
! #  Type
T 1  SDE  0  1  0
T 2  DIF
\end{lstlisting}

\subsubsection{input.TK}

The input.TK file will contain the surface temperatures, in Kelvin, for all surfaces. The file must be named ``input.TK" and be located in the same directory as the MCRT executable.

Using this information, MCRT will calculate a heat balance on all surfaces in the enclosure using the Zonal method which is outlined below. The blackbody emissive power for surface $i$ is given by Equation \ref{eq:bbpower}

\begin{equation}
    E_{b,i} = \sigma T_i^4
    \label{eq:bbpower}
\end{equation}

where, $\sigma$ is the Stefan-Boltzmann constant \SI{5.67E-8}{\watt\per\meter\squared\per\kelvin\tothe{4}} and $T_i$ is the surface temperature in Kelvin.

The surface heat balance for surface $i$ is given in Equation \ref{eq:zonal}.

\begin{equation}
    Q_i = \varepsilon_i E_{b,i} A_i - \sum_{j=1}^N \varepsilon_j E_{b,j} A_j D_{ij}
    \label{eq:zonal}
\end{equation}

An example input.TK file is given below.

\begin{lstlisting}
1  1  300
2  2  300
3  3  300
4  4  300
5  5  300
6  6  300
7  7  300
8  8  300
9  9  300
10 10 300
11 11 300
12 12 300
13 13 300
14 14 300
0  0  0
\end{lstlisting}

As currently programmed, column 1 indicates the lowest surface number in a continuous list of surfaces which contain the same temperature. Column 2 indicates the largest surface number in this same continuous list of surfaces. If, say, for example surfaces 1--5 have a temperature of \SI{200}{\kelvin}, and surfaces 6--14 have a temperature of \SI{300}{\kelvin}, the following input file would also be valid. The last line of the file must be ``0 0 0".

\begin{lstlisting}
1  5  200
6 14  300
0  0  0
\end{lstlisting}

\subsubsection{parameters.txt}

The parameters.txt file will contain two pieces of information. The first line indicates the number of rays to be emitted per surface; the second line indicates whether the simulation is to populate the ``logfile.dat" file which is used by the Ray Tracing Visualization Tool (RTVT) program \footnote{Reference needed.} \footnote{RTVT is a graphical program which will read the logfile.dat file, render a wireframe mesh of the enclosure, and trace the rays from surface-to-surface. This program is not suited to render any more than a few hundred rays per surface and no more than 24--36 surfaces.}. A value of ``1" indicates the file is to be populated; any other value will not populate the logfile.

The the file must be name ``parameters.txt" and be located in the same directory as the MCRT executable. An example input file is shown below, where 1000 rays per surface are to be emitted, and no log file is to be written.

\begin{lstlisting}
1000
2
\end{lstlisting}

\subsection{Vertices}
\label{subsec:verts}

Input for each vertex requires a vertex number and location in an $(x,y,z)$ Cartesian coordinate system, as shown below. The vertices need to be numbered uniquely, with non-negative integer values beginning at 1. Vertices may be entered out of order and the prefix ``V" must be used at the beginning of each vertex line. Currently, the vertex location must be positive values, i.e. remain within quadrant I of a Cartesian coordinate system. All vertex definitions must be entered in the text input file prior to the surface definitions.

\begin{lstlisting}
!   #   x   y   z
V   1   8   0   0
V   2  13   0   0
V   3  13   8   0
V   4   8  13   0
V   5   0  13   0
V   6   0   8   0
V   7   8   8   0
V   8   8   0   5
V   9  13   0   5
V  10  13   8   5
V  11   8  13   5
V  12   0  13   5
V  13   0   8   5
V  14   8   8   5
\end{lstlisting}

\subsection{Surfaces}
\label{subsec:surfs}
Each surface is defined by a number, its respective vertices, and a number of surface properties which are partially an artifact using View3D input files as a starting template. Example surface inputs are shown below. Surfaces must begin with the prefix ``S" and be given a unique non-negative integer surface number beginning at 1.

\begin{lstlisting}
!   #  v1  v2  v3  v4 base cmb  emit   name
S   1   8   9   2   1   0   0   0.90   South-Wall-Bottom-(SDE)
S   2   9  10   3   2   0   0   0.90   East-Wall
S   3  10  11   4   3   0   0   0.99   Angle-Wall
S   4  11  12   5   4   0   0   0.90   North-Wall
S   5  12  13   6   5   0   0   0.90   West-Wall-Top
S   6  13  14   7   6   0   0   0.90   South-Wall-Top
S   7  14   8   1   7   0   0   0.90   West-Wall-Bottom
S   8   1   2   3   7   0   0   0.90   South-Floor
S   9   7   3   4   0   0   8   0.90   Angle-Floor
S  10   4   5   6   7   0   8   0.90   North-Floor
S  11  14  10   9   8   0   0   0.90   South-Ceiling
S  12  14  11  10   0   0   11  0.90   Angle-Ceiling
S  13  13  12  11  14   0   11  0.90   North-Ceiling
\end{lstlisting}

References to the vertex numbers defined previously are given in columns v1--v4. For triangular surfaces, v4 will be ``0". The ``base" column is not used. The ``cmb" column indicates which surface the current surface should be combined with. For example, above surfaces 9 and 10 are to combine with surface 8 to form a single surface, and surfaces 12 and 13 are to combine with surface 11 to form a single surface. The surfaces will be evaluated individually within the MCRT program, but will be combined for reporting purposes at the end of the simulation. ``Emit" is the surface emittance, or emissivity, and the ``name" column indicates the surface name, which is truncated to 10 characters in length.

\subsection{Surface Types}
\label{subsec:surftypes}
Surface types define how each surface is evaluated within MCRT. If the surface definition is not provided, the surface is assumed to be a diffusely emitting surface. Descriptions of the surface type definitions are provided below from \cite{HolmanSpitlerSikha2012}.

\subsubsection{DIF---Diffuse Emitting}
\label{s-s-sec:dif}

\subsubsection{SDE---Specular and Diffuse Emitting}

\subsubsection{SDR---Specular and Diffuse Reflecting}

\subsubsection{DRO---Diffuse Reflection Only}

\subsubsection{SRO---Specular Reflection Only}

\subsection{Comment Lines}
\label{subsec:comments}
Comment lines begin with an ``!" and are not read by the program.

\section{Output}
\label{sec:output}

\subsection{MCOutput.txt}

\subsection{logfile.dat}

\section{Testing}


\section{Future Work}
\label{sec:future work}

The following list outlines the some of the program's deficiencies. Some of these issues are feature driven, i.e.~adding new features will allow the program simulate additional physics. Some of the issues are programming and implementation driven, i.e.~addressing these issues will make better use of the program and make it easier and more intuitive to use.

\subsection{Features}

\begin{itemize}
	\item Reflectance Models:

	\item Participating Media: the current program does not simulate participating media, but this would be an interesting feature to add. A major hurdle that needs to be overcome to solve this problem is that of gridding the domain. Participating media is a volumetric phenomena, and therefore the volume of the enclosure needs to be broken in to smaller blocks which all need to interact with one another. A first pass at accomplishing this task would be to handle rectangular-only enclosures first, then work to develop a more generalized code base.

	\item Adaptive Solution: current program emits the user-specified number at each run. An adaptive solution could be implemented which will run, say, \SI{1E5}{} rays then run another \SI{1E5}{} rays and compare the solutions. If the distribution factors for \SI{1E5}{} is within a user-specified tolerance, the solution could stop there at \SI{2E5}{} rays. If not, another \SI{1E5}{} rays could be run and the check repeated until the solutions converge to within tolerance. The tolerance, step-size, and max/min number of rays should all be user-defined parameters.
\end{itemize}

\subsection{Programming}

\begin{itemize}
    \item Command Line Arguments: the input file names for the programs are currently hard coded. A better, more generalized way would be to pass these to the program as command line arguments. That way, multiple input files with different names could be run in the same directory without needing extra manual or scripting work to move or rename files.

    \item Negative Vertex Coordinates: when negative vertex coordinates are passed to the program, the program crashes.

    \item Heat Balance Inputs: some cleanup could occur for the heat balance inputs, i.e. the input.TK file. First would be to require no ``0 0 0" to indicate the end of the file. The second may be to develop a better input scheme for the data.

    \item Testing: the program only has a limited number of documented tests, none of which occurred recently. This should be integrated with the next bullet point, Continuous Integration. In general, the idea is to develop sets of tests based on independent reference data which are run continuously during the development process to ensure the program works.

    \item Continuous Integration (CI): continuous integration works by developing sets of tests which are run automatically during the development process. Ideally this happens when an incremental change occurs. As of writing this, the MCRT is building on Travis-CI (\url{https://travis-ci.org/BETSRG/MCRT}). Any tests that can be run to verify that the code is working should be added to this. The CI environment can also be leveraged to automatically generate ``release" versions of MCRT. When the ``Create a new release" button on the MCRT GitHub repository has been hit (\url{https://github.com/BETSRG/MCRT/releases/new}), CI can be configured to automatically build the \LaTeX ~documentation, build executables for any Windows, Mac, and Linux platform, and post them to the Release page on the MCRT GitHub repository.

    \item General Code Refactoring: the code relies on global variables everywhere in the code. Some thought should be given to a better code architecture which facilitates code reuse and facilitates program modifications.

    \item RTVT: while not part of MCRT, the program is important to the MCRT project. RTVT currently is limited to between 24--36 surfaces, and should be limited to a few hundred rays per surface. The program was originally written in VB6, which is no longer supported on Windows platforms. Although the program appears to function as basically intended on current Windows 7 \& 10 installations, this is not expected to continue indefinitely. Bringing the code up to be compliant with current VB.net coding practices could be useful. However, if someone is going to go to this level of effort, perhaps a better solution would be to rewrite the program entirely in a cross-platform compatible language such as Python.

\end{itemize}

\newpage
\renewcommand\bibname{References}
\addcontentsline{toc}{section}{References}
\bibliography{../references} %use main dissertation bibtex file
\bibliographystyle{apalike}

\end{document}